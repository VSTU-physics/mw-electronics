\documentclass{hedwork}
\usepackage[utf8]{inputenc}
\usepackage[russian]{babel}
\usepackage[derivative]{hedmaths}
\usepackage{circuitikz}
\begin{document}
\section{Уравнение однородной линии передачи в виде нормальных волн}
Рассмотрим малый участок двухпроводной линии с погонными параметрами
\( L,\ C,\ R,\ G \):
\begin{figure}[h]
    \center
    \begin{circuitikz} \draw
        (0,0) to [R=Rdz] (2,0) to [L=Ldz] (4,0)
        (4,0) -- (7,0)
        (4,0) to [C=Cdz] (4,-3)
        (6,0) to [R=Gdz] (6,-3)
        (0,-3) -- (7,-3)
        (0,0) to [open, v>=$u$] (0, -3)
        (7,-3) to [open, v=$u+du$] (7,0);
    \end{circuitikz}
\end{figure}
Запишем для него законы Кирхгофа:
\begin{equation}
    \left\{
        \begin{array}{l}
            iR + L\pder{i}{t} = -\pder{u}{z},\\
            C\pder{u}{t} + Gu = -\pder{i}{z}.
    \end{array}
    \right.
\end{equation}
Пренебрегая потерями в линии (\( R = G = 0 \)), получим более простую
систему:
\begin{equation}
    \left\{
        \begin{array}{l}
            L\pder{i}{t} = -\pder{u}{z},\\
            C\pder{u}{t} = -\pder{i}{z}.
    \end{array}
    \right.
\end{equation}

Введём поняттие волнового сопротивления линии:
\begin{equation}
    Z = \sqrt{\frac{L}{C}},
\end{equation}
и домножив на него второе уравнение, получим:
\begin{equation}
    \left\{
        \begin{array}{l}
            L\pder{i}{t} = -\pder{u}{z},\\
            ZC\pder{u}{t} = -Z\pder{i}{z}.
    \end{array}
    \right.
\end{equation}
Теперь сложим и вычтем эти уравнения друг из друга:
\begin{equation}
    \left\{
        \begin{array}{l}
            L\pder{i}{t} + ZC\pder{u}{t} = -\pder{u}{z} - Z\pder{i}{z},\\
            L\pder{i}{t} - ZC\pder{u}{t} = -\pder{u}{z} + Z\pder{i}{z}.
    \end{array}
    \right.
\end{equation}
Обозначим \( a_+ = k(u + Zi), a_- = k(u - Zi) \), где \( k \) -- некоторая
постоянная, и перепишем систему в виде:
\begin{equation}
    \left\{
        \begin{array}{l}
            \pder{a_+}{z} = -\sqrt{LC}\pder{a_+}{z},\\
            \pder{a_-}{z} = \sqrt{LC}\pder{a_-}{z},
    \end{array}
    \right.
\end{equation}
откуда ясно, что \( a_+ \) и \( a_- \) -- нормальные волны. Постоянную \( k \)
определим из условия:
\begin{equation}
    P = ui = a_+^2 - a_-^2 = 4k^2uiZ,\quad k = \frac{1}{4\sqrt{Z}}.
\end{equation}
Среди прочих, решениями этой системы являются функции
\begin{equation}
    a_+ = \dot{a}_+e^{i(\omega t - \beta z)} +
          \dot{a}_+^*e^{-i(\omega t - \beta z)},\quad
    a_- = \dot{a}_-e^{i(\omega t + \beta z)} +
          \dot{a}_-^*e^{-i(\omega t + \beta z)},
\end{equation}
где \( \dot{a}_+ \) и \( \dot{a}_- \) -- комплексные амплитуды,
\( \beta = \omega\sqrt{LC} \) -- постоянная распространения.

Определим фазовую и групповвую скорости таких волн:
\begin{equation}
    v_p = \frac{\omega}{\beta} = \sqrt{\frac{1}{LC}},\quad
    v_g = \pder{\omega}{\beta} = v_p\left[1-\beta\pder{v_p}{\omega}\right]^{-1}.
\end{equation}

\section{Уравнение связанных волн в замедляющей структуре}

\end{document}
