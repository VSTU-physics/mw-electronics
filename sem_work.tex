\documentclass{hedwork}
\usepackage[utf8]{inputenc}
\usepackage[russian]{babel}
\usepackage[derivative]{hedmaths}
\usepackage[european resistors, american inductors]{circuitikz}
\begin{document}
\section{Уравнение однородной линии передачи в виде нормальных волн}
Рассмотрим малый участок двухпроводной линии с погонными параметрами
\( L,\ C,\ R,\ G \):
\begin{figure}[h]
    \center
    \begin{circuitikz} \draw
        (0,0) to [R=Rdz] (2,0) to [L=Ldz] (4,0)
        (4,0) -- (8,0)
        (4,0) to [C=Cdz] (4,-3)
        (6,0) to [R=Gdz] (6,-3)
        (0,-3) -- (8,-3)
        (0,0) to [open, v>=$u$] (0, -3)
        (8,-3) to [open, v=$u+du$] (8,0);
    \end{circuitikz}
\end{figure}
Запишем для него законы Кирхгофа:
\begin{equation}
    \left\{
        \begin{array}{l}
            iR + L\pder{i}{t} = -\pder{u}{z},\\
            C\pder{u}{t} + Gu = -\pder{i}{z}.
    \end{array}
    \right.
\end{equation}
Пренебрегая потерями в линии (\( R = G = 0 \)), получим более простую
систему:
\begin{equation}
    \left\{
        \begin{array}{l}
            L\pder{i}{t} = -\pder{u}{z},\\
            C\pder{u}{t} = -\pder{i}{z}.
    \end{array}
    \right.
\end{equation}

Введём поняттие волнового сопротивления линии:
\begin{equation}
    Z = \sqrt{\frac{L}{C}},
\end{equation}
и домножив на него второе уравнение, получим:
\begin{equation}
    \left\{
        \begin{array}{l}
            L\pder{i}{t} = -\pder{u}{z},\\
            ZC\pder{u}{t} = -Z\pder{i}{z}.
    \end{array}
    \right.
\end{equation}
Теперь сложим и вычтем эти уравнения друг из друга:
\begin{equation}
    \left\{
        \begin{array}{l}
            L\pder{i}{t} + ZC\pder{u}{t} = -\pder{u}{z} - Z\pder{i}{z},\\
            L\pder{i}{t} - ZC\pder{u}{t} = -\pder{u}{z} + Z\pder{i}{z}.
    \end{array}
    \right.
\end{equation}
Обозначим \( a_+ = k(u + Zi), a_- = k(u - Zi) \), где \( k \) -- некоторая
постоянная, и перепишем систему в виде:
\begin{equation}
    \left\{
        \begin{array}{l}
            \pder{a_+}{z} = -\sqrt{LC}\pder{a_+}{z},\\
            \pder{a_-}{z} = \sqrt{LC}\pder{a_-}{z},
    \end{array}
    \right.
\end{equation}
откуда ясно, что \( a_+ \) и \( a_- \) -- нормальные волны. Постоянную \( k \)
определим из условия:
\begin{equation}
    P = ui = a_+^2 - a_-^2 = 4k^2uiZ,\quad k = \frac{1}{2\sqrt{Z}}.
\end{equation}
Среди прочих, решениями этой системы являются функции
\begin{equation}
    a_+ = \dot{a}_+e^{i(\omega t - \beta z)} +
          \dot{a}_+^*e^{-i(\omega t - \beta z)},\quad
    a_- = \dot{a}_-e^{i(\omega t + \beta z)} +
          \dot{a}_-^*e^{-i(\omega t + \beta z)},
\end{equation}
где \( \dot{a}_+ \) и \( \dot{a}_- \) -- комплексные амплитуды,
\( \beta = \omega\sqrt{LC} \) -- постоянная распространения.

Определим фазовую и групповвую скорости таких волн:
\begin{equation}
    v_p = \frac{\omega}{\beta} = \sqrt{\frac{1}{LC}},\quad
    v_g = \pder{\omega}{\beta} = v_p\left[1-\beta\pder{v_p}{\omega}\right]^{-1}.
\end{equation}

\section{Уравнение связанных волн в активном ответвителе}
Рассмотрим распространение волн в слабосвязанных структурах без потерь. Пусть в
каждой из структур возбуждается только одна из мод
(назовём их \(a_1\) и \(a_2\), \( \beta_1 \approx \beta_2 \)) и нам известны
погонные коэффициенты связи \( c_{12} \) и \( c_{21} \). Так как мы
рассматриваем активный ответвитель, то волны в структурах бегут в разные
стороны. Система в этом случае имеет вид:

\begin{equation}
    \left\{
    \begin{array}{l}
        \pder{a_1}{z} = -i\beta_1 a_1 + c_{12}a_2,\\
        \pder{a_2}{z} = i\beta_2 a_2 + c_{21}a_1.
    \end{array}
    \right.
\end{equation}

Коэффициент передачи
\begin{equation}
    F_{12} = \left[1 - \frac{(\beta_1 + \beta_2)^2}{4c_{12}c_{21}}\right]^{-1}
    \approx 1.
\end{equation}

Ищем решение системы в виде \( e^{\gamma z} \):
\begin{equation}
    \begin{vmatrix}
        -i\beta_1-\gamma & c_{12} \\
        c_{21} & i\beta_2-\gamma
    \end{vmatrix}
    = 0,
\end{equation}
\begin{equation}
    \gamma_{1,2} = -i\frac{\beta_1-\beta_2}{2} \pm \sqrt{-\frac{(\beta_1 -
    \beta_2)^2}{4} - \beta_1\beta_2 + c_{12}c_{21}}=
    -i\beta \pm \delta,
\end{equation}

\[
    \beta = \frac{\beta_1 - \beta_2}{2},\quad
    \delta = \sqrt{\frac{c_{12}c_{21}}{F_{12}}}.
\]

Таким образом,
\begin{align}
    a_1(z) & = e^{-i\beta z}\left(c_1 e^{-\delta z} + c_2e^{\delta z}\right),\\
    a_2(z) & = \frac{e^{-i\beta z}}{c_{12}}
    \left[(i\beta' - \delta)c_1e^{-\delta z}+
    (i\beta'+\delta)c_2e^{\delta z}\right],\quad
    \beta' = \frac{\beta_1+\beta_2}{2}.
\end{align}

В ответвителе с активной связью \( a_2(l) = 0 \).
\end{document}
